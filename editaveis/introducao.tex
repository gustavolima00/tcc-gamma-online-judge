\chapter[Introdução]{Introdução}

\section{Contextualização}

Maratonas de programação são eventos nos quais programadores testam e aprimoram suas habilidades de programação e resolução de problemas, dentre elas existem vários níveis que podem englobar programadores iniciantes ou mais experientes. Um exemplo de evento dessa natureza é a Maratona de programação UnB, realizada desde 2013 e que acumula até hoje 9 edições. Em geral ela é voltada para o público iniciante e intermediário, pois os problemas variam entre fáceis e difíceis.

Esses eventos foram realizados, na maioria das vezes, internamente sem auxílio de ferramentas e servidores externos, o que dificulta os competidores a encontrar problemas mais antigos de edições passadas, encontrar resolução e tutoriais para as questões, e o mais importante, uma plataforma para testar seus códigos - processo que é fundamental para aprendizagem na área.

Existem vários juízes onlines que fazem o trabalho de armazenar e julgar problemas. No brasil, os mais populares entre competidores são o URI Online Judge, Codeforces e Online Judge. Dentre eles, é possível cadastrar novas questões utilizando o Codeforces ou o URI Online Judge, sendo o URI o único em português. Porém, a indexação dos problemas para buscas, separação em eventos e a escrita de tutoriais não é flexível, e em alguns casos, nem é possível. Além disso, alguns desses juízes possuem problemas de usabilidade e um feedback não assertivo acerca do resultado quando o público são programadores iniciantes, questão apontada também por Francisco (\citeyear{francisco_re:JONEPI}):
\begin{quotation}
    [\dots] Talvez devido à falta de maturidade e à origem baseada em competições, os sistemas apresentam problemas de usabilidade. O processo de submissão de programas ainda apresenta problemas, e o \textit{feedback} não é suficiente para que alunos consigam corrigir muitos dos erros.
\end{quotation}

Conforme se tem observado, a utilização de ferramentas existentes atualmente para o cadastro de questões das edições das Maratonas de programação UnB, dificulta a flexibilização da inovação no ensino de estudantes iniciantes na área.

\section{Justificativa/Motivação}

Os sistemas de juízes onlines tem se popularizado no ensino de programação básica. Apesar disso, essas ferramentas ainda são voltadas para competidores mais experientes e deixam a desejar em relação ao aprendizado para iniciantes. Além disso, disciplinas de programação competitiva e a utilização de juízes online se mostraram eficazes no desenvolvimento de programadores.

Adicionalmente, juízes online não possuem um suporte adequado para o armazenamento e indexação de questões. Em decorrência disso, utilizar juízes online para armazenar questões antigas das maratonas de programação UnB traz pouca flexibilidade, limitando os recursos e impossibilitando o direcionamento da plataforma para estudantes iniciantes.

\section{Objetivos}

Diante do exposto, torna-se objetivo geral do trabalho criar uma plataforma para armazenar e julgar questões das edições anteriores das Maratonas UnB de programação.

E são objetivos específicos: 
\begin{enumerate}
    \item Armazenar questões anteriores das maratonas de programação UnB;
    \item Proporcionar ferramentas de auxílio para programadores iniciantes, como tutoriais e dicas para resolução dos problemas;
    \item Julgar as questões e entregar um feedback assertivo para os usuários a cerca dos códigos enviados à plataforma;
    \item Indexar as questões possibilitar a busca para que o acesso às questões seja mais fácil.
\end{enumerate}

\section{Estrutura do Trabalho}

-

