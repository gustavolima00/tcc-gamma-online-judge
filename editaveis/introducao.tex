\chapter[Introdução]{Introdução}

\section{Contextualização}

O trabalho consiste em um juiz online para armazenar e julgar códigos de questões das edições da
maratonas UnB de programação.
Durante os anos foram realizadas várias edições das maratonas de programação UnB, que são eventos
para competidores testar e aprimorar as suas habilidades de programação e resolução de problemas.
Em geral essas maratonas são interessantes para alunos iniciantes e intermediários em programação
competitiva, pois os problemas variam entre problemas fáceis e mais difíceis. Essas maratonas foram
realizadas internamente, muitas vezes sem auxílio de ferramentas externas e servidores externos, o
que dificulta os competidores a encontrar questões antigas de edições passadas, encontrar resolução
e tutoriais para as questões, e o mais importante, uma plataforma para testar seus códigos,
processo que é fundamental para aprendizagem na área.

Existem vários juízes online que fazem esse trabalho de armazenar e julgar questões, os mais
populares no meio da programação para competição no Brasil são o Online Judge, URI e o Codeforces,
porém o processo para criar eventos e centralizar as questões nesses juízes não é ideal. Dentre
eles só o URI é disponibilizado em português, nele um usuário comum não consegue criar questões
originais, e mesmo criando questões os tutoriais e dicas não são feitos da maneira ideal, as
ferramentas proporcionadas são um uDebug e um fórum para discussão, que ajudam, porém não é ideal,
pois o feedback de um fórum não é imediato, o que pode desencorajar estudantes a procurarem essa
plataforma para esse tipo de uso. O Codeforces é um ótimo juíz e bastante conhecido por ser uma
plataforma mais aberta, nele é possível ver o código de outros competidores em competições
oficiais, após a competição, e nos principais eventos são disponibilizados tutoriais para as
questões, para auxiliar o competidor que teve muita dificuldade e tem curiosidade de saber como é a
resolução de uma questão específica, o que impede a utilização dessa plataforma é que a linguagem
oficial do site é inglês e não é possível criar questões e eventos no site, apenas em grupos, o que
dificulta a catalogação e indexação dos problemas.

O projeto consiste em um repositório que armazenará as questões das edições anteriores das
maratonas de programação, separando as questões por edição das maratonas de programação e
facilitando a busca de questões antigas. O sistema também disponibilizará tags que separam as
questões por tópicos, o que é interessante para treinamento de um tópico específico. A plataforma
disponibilizará os enunciados, dicas e tutoriais de como resolver os problemas e um juíz online que
julgará as soluções dos competidores interessados em treinar suas soluções para essas questões de
edições passadas, com isso o feedback para a resolução dos problemas e busca de soluções será
imediato, o que motiva mais o competidor.


\section{Justificativa/Motivação}

O primeiro contato que eu tive com um juíz online foi no primeiro semestre da faculdade na
disciplina de Algoritmos e Programação para Computadores, nessa disciplina era ensinado programação
básica, e o meu professor utilizava o juíz online URI. Nele o feedback para resolução dos problemas
era imediato, eu sabia quando um código estava errado, lento ou inadequado para aquele problema, o
que facilitava mais no aprendizado pois eu não dependia de um professor corrigir meu código para
saber se estava certo. No terceiro semestre da faculdade tive um contato maior com esse tipo de
Juíz online pois entrei no mundo de programação competitiva, eu competi em várias edições da
Maratona UnB de programação, o que me motivou a participar da seletiva de programação, que me
rendeu uma vaga para competir na Final brasileira de programação em campina grande, lá eu obtive o
26º lugar do brasil em programação. Dentre todas as maratonas que eu fiz eu percebi que o
armazenamento desses eventos geralmente não é feito de forma ideal, com exceção de alguns eventos
como competições da Olimpíada Brasileira de Informática, que tem um site próprio para armazenar
questões de edições passadas, eventos próprios de plataformas, como os contests do Codeforces,
porém muitos eventos não possuem uma plataforma que disponibiliza os enunciados e tutoriais dessas
questões e julgam solução para auxiliar competidores a treinar e aprender mais. Fiz várias das
maratonas UnB de programação e em geral as questões desses eventos são muito boas para iniciantes
aprenderem sobre tópicos clássicos de programação competitiva, acredito que uma plataforma para o
armazenamento dessas questões será uma ótima ferramenta para novos competidores estudarem.

O trabalho a princípio será desenvolvido por mim e futuramente mantido por outras pessoas, visto
isso, a tecnologia utilizada foi Node JS, tanto na interface quanto no servidor para facilitar a
contextualização de contribuidores, e o aprendizado de interessados em contribuir com o projeto,
sendo que na interface é utilizado o React como framework para criação das telas. A linguagem de
programação utilizada tanto na interface, quanto no servidor foi Typescript pelo fato de ser uma
linguagem de programação fortemente tipada, não foi optado a utilização de Javascript, que é uma
linguagem mais comum para a tecnologia, pois apesar do ganho de tempo ao se preocupar menos com a
tipagem, erros futuros podem ser mais difíceis de se encontrar erros no código e revisar códigos de
contribuidores pode ser mais complicado, com uma linguagem fortemente tipada boa parte do processo
de análise é feito pelo compilador, e com um editor de texto configurado erros ficam mais visuais.
O servidor se comunica com um banco de dados orientado a documento, o MongoDB, pois nesse projeto a
flexibilidade do banco em relação ao armazenamento de documento é mais interessante que um banco de
dados relacional. A popularidade entre Node e MongoDB foi outro fator para a escolha dessa
tecnologia.


\section{Objetivos}

\subsection{Objetivo Geral}

Criar plataforma para armazenar e julgar questões das edições anteriores das Maratonas UnB de
programação

\subsection{Objetivos específicos}

Criação de banco de dados para armazenar questões e eventos das maratonas UnB de programação e as
submissões dos usuários
Criação de uma interface que possibilita um usuário encontrar, buscar, ler e enviar questões
Criação de um micro serviço para criar, editar, deletar e fazer busca em questões e eventos. Esse
serviço será utilizado pela interface.
Criação  de um micro serviço para receber códigos dos usuários, julgá-los e retornar o veredito.
Esse serviço será utilizado pela interface.

\section{Estrutura do Trabalho}

-

