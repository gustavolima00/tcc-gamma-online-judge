\chapter{Introdução}

Este capítulo aborda as justificativas e motivações do trabalho. Sua estrutura é composta com pelas seções: 
\nameref{sec:contextualizacao}, que traz um contexto geral para o entendimento do trabalho; 
\nameref{sec:justificativa}, que aborda a justificativa e motivação; 
\nameref{sec:objetivos}, que apresenta os objetivos gerais e específicos;
e, por fim, \nameref{sec:estruturaDoTrabalho} que apresenta a estrutura do trabalho.

\section{Contextualização}
\label{sec:contextualizacao}

Maratonas de programação são competições onde os participantes utilizam conhecimentos de ciência da computação para resolução de problemas. São grupos de competidores, ou competidores individuais, disputando para uma melhor colocação na competição. As equipes recebem problemas e devem resolvê-los utilizando ciência da computação, os problemas são resolvidos utilizando alguma linguagem de programação e julgados por juízes eletrônicos. A colocação dos participantes depende do formato da maratona; um dos formatos de maratona é o ICPC, onde os participantes competem em trios. Esse formato utiliza o número de problemas resolvidos e o tempo como critérios de colocação, assim, os competidores devem resolver o maior número de problemas, o mais rápido possível.

A Maratona UnB de Programação é uma maratona de programação que segue o formato de maratona ICPC, realizada desde 2013, acumula até hoje 10 edições. Seu público vária, não se limitando apenas aos estudantes da UnB. Além disso, o nível dos competidores é abrangente, contendo estudantes iniciantes em programação e estudantes mais experientes.

Outro exemplo é a Olimpíada brasileira de Informática. Diferente da Maratona UnB de Programação, ela é voltada para o público iniciante em programação e a participação é individual. Os níveis dos competidores podem variar entre estudantes do ensino fundamental, ensino médio, técnico ou início de graduação \cite{obi2021:info}.

As Maratonas UnB de Programação foram realizadas, na maioria das edições, sem auxílio de plataformas onlines, dificultando os competidores a encontrar problemas de edições passadas, buscar resolução e tutoriais para as questões e, o mais importante, uma plataforma para testar seus códigos -- processo fundamental para aprendizagem e o desenvolvimento dos competidores.

Existem vários juízes onlines que fazem o trabalho de armazenar e julgar problemas. No Brasil, existem juizes populares como: Beecrowd\footnote{Disponível em: https://www.beecrowd.com.br/}, Codeforces\footnote{Disponível em: https://codeforces.com/} e Online Judge\footnote{Disponível em: https://onlinejudge.org/}. Dentre os citados, é possível cadastrar novas questões utilizando o Codeforces ou o Beecrowd, sendo o Beecrowd o único em português. A indexação dos problemas para buscas, separação em eventos, escrita de tutoriais não é flexível e, em alguns casos, nem é possível. Além disso, alguns desses juízes possuem problemas de usabilidade e um \textit{feedback} não assertivo acerca do resultado, quando o público são programadores iniciantes, questão apontada também por \citeonline{francisco2016juiz}:
\begin{quotation}
     [\dots] Talvez devido à falta de maturidade e à origem baseada em competições, os sistemas apresentam problemas de usabilidade. O processo de submissão de programas ainda apresenta problemas, e o \textit{feedback} não é suficiente para que alunos consigam corrigir muitos dos erros.
\end{quotation}

%% TODO parei aqui
Conforme se tem observado, a utilização de ferramentas existentes atualmente para o cadastro de questões das edições das Maratonas de programação UnB dificulta a indexação dos problemas e limita a flexibilidade. Além disso, é interessante que as questões dos eventos fiquem organizadas em um único local. Essa ideia se assemelha ao site da Olimpíada brasileira de Informática\footnote{Disponível em: https://olimpiada.ic.unicamp.br/}, que concentra as questões e provas das edições anteriores do evento.

\section{Justificativa/Motivação}
\label{sec:justificativa}

Os principais juízes online não possuem um suporte flexível para a indexação de questões das Maratonas UnB de Programação. Além disso, seria interessante a concentração das questões desses eventos em um site próprio, facilitando a busca por problemas.

Atualmente os juízes online tem um formato voltado para competidores de programação. Devido a isso, o retorno das plataformas pode não ser suficiente para alunos iniciantes encontrarem erros presentes em seus códigos \cite{francisco2016juiz}, um fator desmotivante para quem não é experiente e busca uma melhora em programação.

\section{Objetivos}
\label{sec:objetivos}

Diante do exposto, torna-se objetivo do trabalho criar o \textit{Gamma Online Judge} (GOJ). Uma plataforma para armazenar e julgar questões das edições anteriores das Maratonas UnB de programação.

Os objetivos deste trabalho são: 
\begin{enumerate}
    \item armazenar questões anteriores das maratonas de programação UnB;
    \item proporcionar ferramentas de auxílio para programadores iniciantes, como tutoriais e dicas para resolução dos problemas;
    \item julgar as questões e entregar uma resposta aos usuários acerca dos códigos enviados à plataforma;
    \item organizar as questões por eventos possibilitando buscas por questões e eventos na plataforma
\end{enumerate}

\section{Estrutura do Trabalho}
\label{sec:estruturaDoTrabalho}

Este trabalho está estruturado da seguinte forma: na \nameref{chap:ft} são apresentados os conceitos fundamentais para o entendimento do desenvolvimento deste trabalho; na \nameref{cap:metodologia} são definidos os requisitos e a arquitetura do trabalho; os \nameref{cap:resultados} com resultados do desenvolvimento do GOJ; e por fim, a \nameref{chap:conclusao}, que apresenta as considerações finais do trabalho.
