\chapter[Considerações finais]{Considerações finais}
\label{chap:conclusao}

Conforme visto ao longo do trabalho, juízes onlines são ferramentas importantes para treinos de programação. Eles possuem problemas que podem ser oriundos de maratonas de programação, e assim, permitem que os usuários testem suas soluções.

Existem diversos juízes disponíveis para treinos de programação. Alguns deles foram analisados no levantamento realizado, o que permitiu identificar a viabilidade das plataformas e as funcionalidades disponíveis. Porém, como observado na subseção \ref{subsec:funcionalidades}, nenhum dos juízes da amostra possuía todas as funcionalidades consideradas relevantes para um projeto como o Gamma Online Jugde.

Então, objetivando reunir as questões das maratonas UnB de programação, o GOJ permite que os usuários acessem os problemas de eventos passados, e testem suas soluções para tais problemas. Suas funcionalidades incluem: a separação das questões por eventos, uma plataforma em português, o envio dos problemas das maratonas, a adição de tutorias nos problemas - funcionalidades estas não reunidas nos juízes levantados. 

Desse modo, o novo juiz online criado atende todos os requisitos do projeto, e preenche, de certo modo, uma lacuna existente na seara dos juízes online. 

Ademais, como mostra a seção \ref{sec:modulosDaAplicacao}, a aplicação foi dividida em módulos, contendo um juiz eletrônico, uma API, uma interface de usuário e uma interface para os administradores, corroborando o atendimento dos requisitos e possibilitando o uso eficiente da plataforma por usuários e administradores.

Com isso, conclui-se que o juiz criado atende os objetivos e consegue armazenar as questões passadas das maratonas UnB de programação. As questões são divididas por eventos, possuem tutoriais e todo o projeto é open source, melhorando a possibilidade de suporte de terceiros.