\chapter[Cronograma]{Cronograma}
\label{chap:cronograma}

Neste capítulo será abordado as funcionalidades que ficaram pendentes para a entrega da aplicação e um prazo estimado para que as funcionalidades aqui descritas sejam completas; ele é composto pelas seções:
\nameref{sec:etapasPendentes}, que descreve as etapas seguintes de desenvolvimento do projeto.
E por fim \nameref{sec:tempoDeExecucao}, que traz estimativas sobre o tempo para o desenvolvimento das etapas pendentes.

\section{Etapas pendentes}
\label{sec:etapasPendentes}

Algumas etapas do trabalho já foram concluídas, deixando funcionalidades pendentes a serem desenvolvidas. As etapas desenvolvidas foram descritas no Capítulo \ref{cap:resultados}, elas envolvem uma API para o armazenamento dos dados e a interface da aplicação.

As funcionalidades pendentes estão relacionadas ao módulo do juiz das questões (Capítulo \ref{cap:metodologia}). Esse módulo se assemelha bastante ao BOCA \cite{de2004boca}, e é bastante provável que no desenvolvimento esse sistema seja usado como base. Visto isso, no cronograma o tempo para o entendimento do sistema BOCA e análise da viabilidade de aproveita-lo é importante.

A etapa do desenvolvimento deve levar em conta que envio e processamento dos arquivos pode ser perigoso ao sistema. Por se tratar de códigos que podem ter conteúdo malicioso, o sistema \textbf{deve} prever essa possibilidade e se adequar para que a execução de um código externo não afete sua integridade. No final do desenvolvimento é necessário fazer a integração do sistema com as funcionalidades já existentes, após isso é esperado que a aplicação funcione por completo

\section{Tempo de execução}
\label{sec:tempoDeExecucao}

A seguir está representada uma lista com os tempos estimados para execução das etapas discutidas na Seção \ref{sec:etapasPendentes}:

\begin{itemize}
    \item Semana 1-2 - Análise de sistemas similares. Levantamento de requisitos e construção da arquitetura.
    \item Semana 3-7 - Desenvolvimento da aplicação 
    \item Semana 7-10 - Integração das funcionalidades desenvolvidas com o resto do sistema
\end{itemize}

Vale ressaltar que os dados aqui presentes são estimativos. O tempo pode variar para mais ou para menos. As informações ficam mais assertivas com ao decorrer do desenvolvimento.